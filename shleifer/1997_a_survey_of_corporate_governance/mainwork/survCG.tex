
\section{Introduction}

\subsection{Preview}
	\begin{frame}{Sum}
	This survey deals with the \textbf{separation of financing and management of firms}, and tries to discuss \textbf{how to deal with} this separation in theory and in practice. 
\end{frame}

	\begin{frame}{Questions}
	\textbf{C}orporate \textbf{G}overnance deals with the ways in which suppliers of finance to corporations assure themselves of getting a return on their investments.\\Related problems:
	\begin{itemize}
		\item Repartriation of profits
		\item Agency problems
		\item Control power of investors over managers
	\end{itemize}

\end{frame}
%%-----------------------------authors
	\begin{frame}{
		\sc{Andrei Shleifer}}
	\begin{wrapfigure}{0}[0pt]{0.35\textwidth}%{行数}[位置][超出长度]{宽度}
		\includegraphics[height=2.8cm]{grap/shleifer.jpg}
	\end{wrapfigure}
	Professor of Economics at Harvard University, Andrei Shleifer holds an undergraduate degree from Harvard and a Ph.D. from MIT.\\
	\vspace{1em}
	Shleifer has worked in the areas of comparative corporate governance, law and finance, behavioral finance, as well as institutional economics(制度经济学). According to RePEc, Shleifer is the most cited economist in the world.
\end{frame}

	\begin{frame}{
	\sc{Robert W. Vishny}}%...少了个},qnmlgb。。。

	\includegraphics[height=3cm]{grap/vishny.jpg}

	Robert Vishny’s area of expertise is behavioral and institutional finance. Amongst the areas that Vishny studied in the past were the market for corporate control, corporate governance around the world, privatization and the role of government in the economy,investor sentiment, and the limits of arbitrage.
\end{frame}

	\begin{frame}{Disagreement}
		Disagreement on existing governance mechanism:
		\begin{itemize}
			\item The US mechanism is nice
			\item More highly leveraged organizations, LBOs
			\item Replace Anglo-Saxon CG mechanism with German\&Japan pattern
		\end{itemize}
\end{frame}

\begin{frame}
	\begin{itemize}
		\item "In the long-run, product mkt competition would force firms to minimize costs, including CG mechanisms. Competition would take care of CG."
		\item MKT competition can't explain in actual practice.\\
	
	\end{itemize}
	\onslide<3-> Because production capital is \textbf{specific} and \textbf{sunk}, and entrepreneurs cannot rent it every minute. 因此就算外部投资者不参与公司生产经营的决策,尽管投资者只是借出资本,他有权为他的沉没成本要求一定的补偿。\\
	(\textit{The people who sink the capital \textbf{need to be offset} by return on this capital.})
\end{frame}

	\begin{frame}{Selection of Empirical Evidence}
	\onslide<1-> US, Japan, Germany, Italy, Sweden, and Russia received many attention.\par
	\onslide<2-> 作者认为英美、日、德的公司治理系统是世界上最好的几种模式,因此本文用了几个section去论述它们之间的微妙区别。\par
	\onslide<3-> 贡献:There is little systematic research on Russia's 
	\textbf{C}orporate \textbf{G}overnance, and there are little research on \textbf{C}orporate \textbf{G}overnance around the world. 
	
\end{frame}

	\begin{frame}{Research Objective}
		Two aspects:
		\begin{enumerate}
		\item CG: Agency perspective, sometimes refered to as separation of ownership and control.
		\item How investors get the managers to give them back their money. 
	\end{enumerate}
\end{frame}
		
\subsection{Related Topics of CG}
	\begin{frame}{Related yet excluded Topics}
	Topics closely related to CG, while this article dose not deal with them.
	\begin{itemize}
		\item Foundations of contract theory 
		\item Basic elements of Firm Theory:\\
		 make/buy decision(零部件自制或外购的决策), vertical integration(纵向合并)
		\item Noncapitalist ownership patterns: \\
		 worker ownership, nonprofit organizations
		\item Financial intermediaries: function as collecting savings from the public ignored.
	\end{itemize}
	
\end{frame}

%-----------------------introduction done


\section{Agency Problem}
\subsection{Nature of Agency Problems}
	\begin{frame}{Contracts}
		\begin{itemize}
			\item Agency problem is an essential element of contractual view of firm.
			\item The essence of agency problem is separation of ownership and control.
			\item 投资者为了确保自己的资金没有被私自挪用或浪费在不盈利的项目上,一般而言,他们就与经理签订合同,限定他们对资金的使用范围、限定最后的利润如何分配。但是由于合同设计不可能是complete,经理和投资者就需要分配剩余控制权(the rights to make decisions where contract dosen't forsee.)
		\end{itemize}

	\end{frame}

\begin{frame}
		\begin{itemize}
			\item 由于投资者不具备足够的知识和信息去决定所有的意外情况,因此经理通常有大量剩余控制权,能自由决定如何支配资金。
			\item Business judgment rule keeps the court out of the affairs of companies.
			\item Free-rider: small shareholders have too little info. to excercise the control rights they have.
		\end{itemize}

		\vspace{2em}
		\onslide<4-> In result, the effective control right of managers end up being much more extensive than they would have been if courts or financiers became actively involved in detailed contract enforcement.

\end{frame}
	
	\begin{frame}{Management Discretion(自行决定的自由)}
		\begin{itemize}
			\item Expropriate: take the cash out; transfer pricing(经理自己建立一个小公司,把公司的产品低价卖给自己的公司); sell the assets(eg将子公司或资产低价或高价卖给其他公司)
			\item Law: 许多国家的投资者保护法增加经理在职消费、挪用公款的成本。但是有些国家法律保护并不健全,经理可能会肆无忌惮地假公济私,损害公司价值。In many countries today, law protect investors better than it dose in Russia, Korea, or Italy.
			\item Staying on the job even they are no longer competant or qualified to run the firm(Shleifer and Vishny(1989)). This is a costliest manifestation of the agency problem(Jesen and Ruback(1983)). 
		\end{itemize}
\end{frame}
	
\begin{frame}
		\begin{itemize}
			\item Managerial Opportunism: Expropriation of investors and misallocation of company funds reduce the amount of resources that investors are willing to put up ex ante to finance the firm(Williamson(1985),Grossman and Hart(1986)). 
			\item Efficiency of the ex post resourcce allocation: 当经理不持有公司的股票时,对于获取的资金,经理有一定自由可以选择投资什么项目。他可以选择使自己的私人利益更大、但投资者无法收回沉没成本的项目。
			\item Corruption: 纵使投资者可以通过用损失的一部分钱去贿赂经理不采取上面那个项目,但实际中这种贿赂很少,因为需要获得大多数股东同意,但经理个人只需要同意少数董事给他的贿赂。
		\end{itemize}
\end{frame}

\begin{frame}

	\onslide<1-> "Managers' legal duty of loyalty to shareholders": 尽管贿赂能改善ex post的资源分配,但是\\
	\onslide<2-> \textit{the public dose not want to give the bureaucrats incentives to come up with ever increasing obstacles to private activity solely to create corruption opportunities(Shleifer and Vishny(1993)).}\\
			公众不愿意因为腐败而增加官僚为私有部门活动设置障碍的机会
\end{frame}

\subsection{Incentive Contract}
	\begin{frame}{Use incentive contract to solve the problems}
		Grant the manager a highly contingent, long term incentive contract ex ante to \textbf{align his interests} with those of investors. \\
		\onslide<2-> Optimal incentive contract is determined by\\
		\begin{itemize}
		 \item manager's risk aver., \\
		 \item the importance of his decisions, \\
		 \item his ability to pay for the cash flow ownership up front(帮助投资者经营公司、获利的能力)\\
		\vspace{1em}
		\item share ownership
		\item stock options
 		\item a threat to dismissal if income is low
		\end{itemize}

\end{frame}

	\begin{frame}{Problems with incentive contracts}
	\begin{itemize}
	\item Management ownership in large firms is too small to make managers intersted in profit maximization.
	\item High powered incentive contract create enormouse opportunities for self-dealing(假公济私) for the managers. (Yermack(1997)发现经理会在刚发布好消息前获得期权,在发布坏消息后推迟获得期权)
	
	\end{itemize}

\end{frame}

\subsection{Evidence on Agency Costs}

	\begin{frame}{Managerial Behavior}
		Managerial investment decisions may reflect their personal intersts rather than those of the investors.
		\begin{itemize}
		\item Managers choose to reinvest the free cash rather than return it to investors: \\
		oil industry, mid 1980s.花了\$20 per barrel去探测新石油,而不是将利润反还给投资者,甚至也没有去购买当时市面上仅售\$6 per barrel的石油(Jesen(1986))。
		\item Aquisition decisions: \\
			Lewellen, Loderer, and Rosenfeld(1985) find that negative returns are most common for bidders in which their managers hold little equity
	\end{itemize}	
\end{frame}			
			
		\begin{frame}
		\begin{itemize}
		\item Management resistance to takeovers: when managers take anti-takeover actions, shareholders lose. 一些证据表明,经理拒绝公司被兼并通常是为了维护他的私人利益,而不是为了股东\\
		比如毒药丸计划会损害股东利益(Ryngaert(1988), Malatesta and Walking(1988))
		\item a less direct evidence:\\
			Johnson, Magee, Nagarajan, and Newman(1985)发现经理的突然死亡(飞机失事、心脏病etc.)会增加股价。股价增加最多的是建立了庞大商业帝国并且对股东回报很少的企业。证明了随着掌权的经理逝世,控制权带来的私利也消失了,公司的价值得到了提升。
		\item control is valued:\\
			Compare the prices of shares with identical divident rights, but differential voting rights. Some studies show that in th US, shares with superior voting rights trade at a premium. 证明控制权本身是有一定价值,掌握控制权的管理层能得到少数股东得不到的收益。
		\end{itemize}

\end{frame}

%---------------------
\section{Financing Without Governance}%How f raise mony without giving out power

\subsection{Reputation-building}
	\begin{frame}{Why investors entrust capital to firms}

尽管有这么多代理问题,投资者并不能完全保障他的投资能有回报,为什么外部投资仍然存在,出现在了大多数经济体中,在发达经济体中也大范围存在?
\end{frame}

	\begin{frame}
There are two explainations:
		\begin{itemize}
		\item Firm and managers have reputations:\\
		经理为了在将来继续获取投资,就需要构建良好的声誉去说服投资者。否则,如果投资者预期这个公司的经理会侵占他的利益,投资者事前就不会去投资这家公司。
		\item Excessive investor optimism: Investors are gullible and get taken(过度乐观,轻信,被利用)\\
		投资者可能只对短期股票增值感兴趣,没有想太多最后能否收回本金。
		\item 极端例子:Ponzi scheme(借新的债去还旧债), Pyramid scheme(层压式传销,金字塔骗局)
		\end{itemize}

\end{frame}


%
\section{Approaches to CG}
%3,Legal Protection

%
\begin{frame}{Approaches to Corporate Governance}
	There are two approaches to assure that investors get their returns back.
\begin{itemize}
	\item Legal protection
	\item Large Investors
\end{itemize}

\end{frame}
%
\subsection{Legal Protection}%give investors power through legal protection
	\begin{frame}{Legal Protection}
	\begin{itemize}
	\item 1 Contract: 外部投资相当于投资者与企业间\textbf{缔结契约},如果经理违反了合同条款,投资者有权告到法院,通过法律途径强制执行他的权利。
	\item 2 股东一个很重要的法律权利是对公司事务的\textbf{投票权}:\\
	比如对并购、清算、董事会选举
	\item 但是投票权通常很难实行,实行成本很高。因为在很多国家股东不能通过邮件投票,必须亲自去股东大会投票。而且他们拥有的信息不如管理层。尽管在发达国家有法律保障投票的举行,但经理常常参与投票过程,私下争取股东的支持。
\end{itemize}
\end{frame}
	
	\begin{frame}{}
	\begin{itemize}
	\item 尽管董事会是股东选出来的,董事会并不一定代表股东的利益\\
	\item structure of corpor. boards: \\
	德国双层董事会、日本内部人控制的董事会、美国折中型董事会(Charkham(1994))\\
	\item 公司的经营出现重大灾难后董事会才开始行动(Warner, Watts, and Wruck(1988))\\
	\item 日德的证据表明除了极端情况,董事会都处于十分被动的地位(Kaplan(1994a,b))\\
	\item 美国公司的董事会一般都在管理层的俘获下(Mace(1971),Jesen(1993))
\end{itemize}
\end{frame}
	

	\begin{frame}{}
	\begin{itemize}
	\item 3 \textbf{duty of loyalty} of the managers to the shareholders:
	\item  OECD国家的法院广泛承认经理对股东有尽忠的义务,因为股东的大部分投资都成为了沉没成本,无法收回,而且与雇员不同,短期没有回报,而且股东在面临资产的征用(expropriation)方面得不到很好的保护。\\
	为了激励股东在事前拿出投资,他们需要更强的保护,比如“尽忠的义务”。
	\item 这个义务的主要来自法律对经理假公济私的惩罚;有些国家的法律直接禁止经理假公济私,或者规定经理在决定重大事务前必须通过董事会的同意。
	\item 不同国家的法院对“尽忠的义务”的执行各不相同。美国法院会惩罚经理的偷盗渎职,但是不会管他们奖励自己超额工资(Excessive pay)的行为。
\end{itemize}
\end{frame}

	\begin{frame}{}
	\begin{itemize}
	\item 4 Legal protection on creditors:\\
	违约或破产时,债权人可以索取作为抵押品的资产,对是否重组公司有投票权,有权更换重组公司的经理。\\
	但是对于债权人,不同债权人利益相冲突,收回破产公司的资产也很困难。
	\item 作者认为\\在美国、日、德,法律保护尚且比较完善,法律能保护部分投资者的权利,法院能较好地执行法律,但仍留给经理不少假公济私的余地。在其它国家情况法律更加薄弱,法律本身并不能充分保障投资者的利益。在这些国家公司治理主要不是靠法律保护。

\end{itemize}
\end{frame}

%4,
\subsection[Concentrated Ownership]{Ownership by Large Investors}
%large shareholdings, relationship banking, takeovers--can be viewed as examples of large investors excercising their power.
%\subsection{Agency Costs}
	\begin{frame}{Ownership by Large Investor}
	法律并不能充分保障small investors的控制权让他们付出自己的资金, 也许投资者可以通过规模变大来得到更有效的控制权,主要形式是大股东、大收购者、大债权人。文章的这个section讨论的是集中的所有权及其成本。
\end{frame}

	\begin{frame}{Large Shareholders}
	几个不同国家的股东结构
	\begin{itemize}
		\item 占有超过51\%股份的大股东拥有足够的控制权、有足够动机去收集信息监督经理人。但是在美国,大股东是不常见的。
		\item 也许是因为法律限制很高的所有权,还存在银行、基金、保险公司等其他机构的控制权(Roe(1994))
		\item 英国的rule规定必须分散股权
		\item 德国企业通常有股权很大的股东,由银行或者家族持有,有约80\%的企业有超过25\%股权的非银行大股东。还产生了用金字塔层级控股很多子公司的情况
		\item 日本股权没有德国那样集中,但是一般有交叉持股、银行持股的情况。
		\item 法国常有cross ownership and core investors
		\item 在大部分欧洲、拉丁美洲、东亚、非洲,时常是由创始人完全控制。
	\end{itemize}
\end{frame}

	\begin{frame}
	一些证据表明大股东在CG上有积极作用
	\begin{itemize}
		\item 在德国,有大股东的企业有higher turnover of directors
		\item 在日本,有大股东的企业比没有大股东的企业更可能在公司经营不善的时候替换经理层
		\item 在日本,大股东能减少经理能自由处置的支出:广告费、R\%D、娱乐支出
		\item 美国,大的外部股东能增加企业被收购的可能性
	\end{itemize}
	\onslide<5-> 但是大股东的投票权的实施还是取决于当地的法律。大股东机制的有效性与他们能否保障自己的权利紧密相关。\par
	\onslide<6-> 极端例子,在Russia,一个外国人要控制俄罗斯的企业至少需要75\%的股份,但一个俄罗斯人只需要25\%。俄罗斯公司的经理可能会通过一系列手段愚弄外国投资者,比如通告外国股东持有的是非法股份、遗失了投票记录等等,而当地法律对此无能为力。
\end{frame}	

%
\begin{frame}{Large Takeovers}
	takeover可以被看做是一种股权集中,因为在收购过程中发起的要约收购可以集中股东分散的股权\par
	\onslide<2-> 一些证据表明收购可以缓解公司治理问题
	\begin{itemize}
	\item 收购增加了并购方和目标公司的combined value,意味着预期利润最后会增加(Jesen and Ruback(1983))
	\item 目标公司通常是经营不善(poor performing)的公司(Palepu(1985))。而且在收购后,经理通常会被换掉
	\item 收购还能解决现金流问题。收购通常使得公司利润能够分配给投资者
	\end{itemize}

\end{frame}

\begin{frame}{收购机制的有效性和成本}
	
	\begin{itemize}
	\item 收购通常成本高昂:并购方为了取得控制权,通常要付给目标公司股东预期收益的溢价
	\item 并购实际上会从增加并购方代理成本:当并购方经理为取得目标公司的控制权(这是一种private benefit of control)而为并购案多付钱(overpay)时,就损害了公司股东的利益(Shleifer and Vishny(1988))
	\item 并购需要流动的资本市场:80年代美国杠杆并购浪潮中许多公司就利用了大量垃圾债券融资
	\item 法律因素:当恶意并购不再被法律允许,政策上的压力终止了美国80年代的并购浪潮(Jesen(1993))
	\end{itemize}

\end{frame}

%
	\begin{frame}{Large Creditors}
	\begin{itemize}
	\item 与大股东一样,大债权人也对企业进行了大量投资,希望从投资中得到收益
	\item 债权人的权利来源于1. 对公司破产或违约时的控制权;2. 每一期债权人都有权得到利息,分走一部分现金流权
	\item 许多国家,银行会同时持有公司的股票和债券,使得大债权人与大股东的性质相近
	\end{itemize}

\end{frame}

	\begin{frame}{Empirical Evidence}
	一些经验证据反映了大债权人的作用
	\begin{itemize}
	\item 在日本,与银行有债务关系的公司在经营不善时,经理更容易被替换
	\item 对1974年的德国,相比其他所有者,银行改善公司经营(company performance)的程度最大
	\item 在美国20世纪早期,J.P.Morgan摩根银行所投资的公司中,JPM对公司治理有重要作用
	\end{itemize}
\end{frame}

	\begin{frame}{Effectiveness of Large Creditors}
	
	\begin{itemize}
	\item 和大股东机制一样,大债权人的有效性取决于法律保护程度
	\item 比如德国和日本,银行对企业有重要治理作用,因为银行有重要投票权、坐在董事会里、在借贷方面处于领导地位、还可以为债权人提供友好的法律环境
	\item 但在意大利,将控制权转移给银行的法律程序不完善,使得银行治理的有效性较低(Barca(1995))
	\end{itemize}
	\vspace{1em}
	\onslide<4-> 因此相比美国,世界上很多其他法律比较薄弱的国家都采取large investor的方法参与公司事务。因为大投资者负担比分散的投资者更少,更容易不依靠法律途径实行他们的权利。

\end{frame}

%5,
\subsection{Costs of Large Investors}
%
\begin{frame}{Costs of Large Investors}
	大投资者机制产生成本的原因
	\begin{enumerate}
	\item 集中的所有权不利于大投资者\textbf{分散风险}
	\item 大投资者通常\textbf{代表他们自己的利益},可能与公司的其他投资者、雇员、经理有冲突。大投资者可能侵占其他人的利益来扩大自己的财富
	\end{enumerate}
\end{frame}

\begin{frame}{Costs of Large Investors}
	
	主要有以下这些cost:
	\begin{itemize}
		\item 直接侵占其他投资者、经理、雇员的利益:\\
		eg.通过给他们自己特殊的分红、与另一个公司关联交易
		\item 为了个人目标侵害公司利益:\\
		eg.和代理问题中的激励一样,公司经营最开始会随着大股东、经理股份的增加而改善,但大股东过高的股权可能会被用来谋私利,这就会产生成本(Morck \textit{et al.})
		\item 对其他stakeholders的负激励:\\
		eg.大投资者对资产的侵占(expropriation)会对雇员、经理产生一个负向激励。当雇员被严密监督或者很容易被解雇时,他们可能会减少自己的工作努力。\\
		eg.对其他投资者的负激励导致他们可能会减少外部投资
		\end{itemize}
\end{frame}


%6,specific governance arrangement
\section{CG Mechanisms}
%contract mechanisms for agency problems
	\begin{frame}{Corporate Governance Mechanisms}
	这一节讨论的是债券、股票,以及国有产权等参与公司治理的不同方式,从契约机制的视角解释一些代理问题。

\end{frame}

\subsection{Debt G \& Equity G}
	\begin{frame}{}
	这一小节讨论的是债权人和股东对比\par
	在有大投资人时,债权人和股东的相对权威需要结合legal protection、ease of ownership concentration来讨论。
		\begin{itemize}
		\item 目前世界上主要的借贷形式是银行贷款,银行通常是大投资者。然而美国、加拿大、英国的债权通常比较分散,尽管比较债务并不集中,但有为债权人提供的法律保护仍然比分散的股东效率高,因为一旦公司违约,债权人有权索取抵押物、或投票将公司重组、投票更换重组公司的经理。
		\item 不同于股票,当借贷十分分散时也许会更加tough。比如大量小债主的债务被违约了,和这些债权人的协商会比单独大债权人更困难,小债主们可能会迫使公司破产。
		\item 不同于债权人,个人股东的投资没有被许诺任何形式的回报,甚至没有被许诺何时公司能变现。尽管他们时常得到分红
	\end{itemize}
\end{frame}

	\begin{frame}{}
	\begin{itemize}
		\item 股东最主要的权利是投票选举董事会。股东的投票权依赖于当地法律保护程度,但许多小股东甚至没有激励去得知如何投票。只有当投票权比较集中时,投票权才十分有价值,比如要约收购或大量持股。因此在这一方面,集中的股权比集中的债权更有权威。
		\item 一个现象:年轻公司通常没有很有价值的抵押品,通常进行股权融资。而成熟公司或成熟的经济体在外部融资方面通常利用银行贷款。
		\item 在少数投资者完全没有法律保护时,为什么公司仍然能通过股票融资:\\在经济快速增长阶段,短期内大投资者可以信赖公司的名誉reputation和短期快速增长的回报;\\
		还有一个可能是市场有大量泡沫、投资者过度乐观
		\end{itemize}

\end{frame}

%--------------------------
\subsection[LBO]{Leveraged Buy Outs}
	\begin{frame}{LBO(杠杆收购)}
	在美国八九十年代,杠杆收购是一个很流行的现象,同时反映了大投资者的收益和成本。杠杆收购就是公司或个体利用收购目标的资产作为债务抵押,收购另一家公司的策略。\par
	\onslide<2-> 通常是利用专门成立的控股收购公司、银行、募集贷款。通常控股收购公司只出小部分的钱,资金大部分来自银行抵押借款、机构借款和发行垃圾债券\par
	\onslide<3-> 相当于通过了垃圾债券将投资者们聚集起来,于是投资公司的经理就通过杠杆收购形成了一个很集中的所有权,这个所有权属于LBO基金会、提供贷款的银行、还有普通的债券持有者。
\end{frame}
\begin{frame}{Benifits and Costs of LBOs}
	%\onslide<>
	Benifits:类似大投资者,LBO也是一个有效率的组织
	\begin{itemize}
		\item LBO通常会以一定溢价购买旧股东的股票,这意味着至少他预期未来公司利润会比现在高
		\item LBO之后一些公司成功上市,实证研究发现它们的确实利润上升了
		\item 一些证据表明利润上升是由于代理成本减少了
	\end{itemize}
	\onslide<4-> Costs of heavily concentrated ownership
	\begin{itemize}
		\item 投资者严密的监督可能会阻碍公司未来的投资和发展
		\item 证据表明LBO并不是一个可以永久维持的组织,一般投资公司五六年内就会将资产卖出
		\item 就算是那些成功上市的LBO公司,它们在债务和股权方面都会非常集中
	\end{itemize}
\end{frame}

\subsection[SOE]{Cooperatives \& State Ownership}%a manifestation of a radical failure of CG
	\begin{frame}{}
	%\onslide<>
	有些公共品,例如health care, child care, railroads, electricity, police, prison, schools需要由合作社或者政府来提供,这些东西是追求利润最大化的企业所办不到的。\par
	\onslide<2-> Costs
	\begin{itemize}
	\item 负外部性:污染问题最严重的是former communist countries的国企
	\item 生产:国企效率低下,损害国家的财富,因为它们常常无视社会需求
	\item 管理人员:官员对国企有完全集中的控制权,但没有现金流权,利润属于国家;官员们追求的政治目标有可能损害社会福利,因此国企的效率一般很低
	\end{itemize}
	\onslide<6-> 尽管Russia进行了私有化,但缺少少数股东保护法,经理同时占有控制权和现金流权,并且享受控制权带来的利益,使得代理问题十分严重。
\end{frame}

%

\section{Which system is the best}%a good CG system should commbine some type of large investors with legal protection of both their rights and those of small investors. US, Germany, Japan
\begin{frame}{Legal Protection and Large Investors}
现今,英美两国公司治理机制主要依赖法律保护,除了收购过程,不太依赖大投资者;在欧洲大陆和日本主要依赖大投资者和银行;在世界其他国家,很多都依赖家族控制,少部分依赖大投资者和银行。\par
\onslide<2-> 本文的结论是:对投资者的法律保护和一些形式的所有权集中是公司治理机制的重要元素。\par
\onslide<3-> 分析了美国、德国、日本的所有权分配情况,证据表明这些发达国家至少对部分类型投资者提供了有效的法律保护,同时有大投资者巩固公司治理机制。\par
\onslide<4-> 然后对比英美的例子,分析了意大利、印度、拉丁美洲的例子,说明这些国家因为法律保护薄弱,大部分是家族控股,只有少量的外部融资和外部公司治理,最后导致治理机制的效率很低。
\end{frame}

\begin{frame}{Which is the best}
	%\onslide<>
	\begin{itemize}
	\item 美国、德国、和日本的法律保护
	\item 证据表明,在美国政治因素比经济因素更多地影响了投资者保护法的订立(Roe(1994))
	\item 美国的政治法律更多地保护小股东、时常使大投资者退却,使得公司治理机制并不是最有效率的:比如对80年代的收购热潮的禁令,antilarge-shareholders policies
	\item 日本德国同样不是最有效率的,有影响力的银行会阻止一些投资者保护法的订立,比如disclosure rule, prohibitions on insider trading, 少数股东保护法。日德的法律服务于影响很大的投资者、银行
	\end{itemize}

\end{frame}

\begin{frame}{What kind of large invetor is better}
	美国的takeover,德国的银行长期大量持股,OR日本混合了公开债券市场?\\
	No answer from available research.
	\begin{itemize}
		\item 支持美国模式的人认为投资者法律保护更有效,同时takeover的存在补充了这一机制。
		\item 支持日德模式的证据表明,在日本,与银行有关系的企业经济压力更小,更容易获得融资。有许多文献表明美国公司治理机制,特别是takeover,使得公司经理的行为特别短视,降低了投资效率。
	\end{itemize}
\end{frame}

\begin{frame}{What kind of large invetor is better}
\begin{itemize}
		\item 最后作者论述道,尽管80年代的收购狂潮被法律终止了,但新的所有权集中机制仍在不断出现,而且为小投资者提供的法律保护使美国的年轻公司比其他国家的年轻公司能更容易得到融资。他认为很难说美国的CG模式不好
		\item 作者认为日德模式中,长期投资者能很耐心的参与公司治理,有足够的信息去帮助公司缓解压力。但投资者在CG中可能不够有效率,比如银行可能花更多精力去处理自身内部的代理问题,而不怎么参与公司治理。
		\item 作者的结论是,这些国家的治理模式各有各的好处和坏处,现有的研究还不能证明哪个模式最好。
	\end{itemize}
\end{frame}



%---------------------done
\section{Summary}
\subsection{Main Conclusion}
\begin{frame}{Summary}
	\begin{itemize}
	\item 本文叙述了公司治理的框架。公司治理就是处理一个两权分离的代理问题,公司治理一个最基本的问题是如何保证投资者能从投资中获取收益。
	\item 其次,描述了几个公司治理的方法:\\
	投资者基于公司的口碑去投资;或者依赖于投资者对未来能收回本金的乐观预期;依赖法律保护;集中所有权。\\
	\item 大投资者为公司治理带来好处的同时也会产生成本:尽管大投资者能解决部分代理问题,但他也会侵占其他投资者利益,增加其他方面的成本。
	\item 现有证据并不能证明哪种治理模式是最好的。
	\end{itemize}

\end{frame}

\subsection{Unresolved Puzzles}
\begin{frame}{Puzzles}

	\begin{itemize}
	\item 考虑到高管对公司价值的重要影响,为什么美国和世界其他地方没有采取非常强力的激励合同?
	\item 其他国家投资者保护法律的性质是怎样的?文献较少
	\item 集中所有权的成本和收益有没有重要影响:大投资者对经理的限制究竟够不够强(tough)?
	\item 为什么发展中国家缺少少数投资者保护法(minority protection),却仍然有分散的股东买走股票?
	\item 世界其他国家的政治和经济究竟对公司治理效率有怎样的影响?
	\end{itemize}

\end{frame}