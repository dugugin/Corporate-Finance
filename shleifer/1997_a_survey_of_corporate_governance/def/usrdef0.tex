%===================================
%% Self defined command
%===================final options====================

% ------------------设置用acrobat打开就会全屏显示,在sty文件里已经设置了  
%\hypersetup{pdfpagemode=FullScreen}

%--------------------是否逐条显示--------------------
%\beamerdefaultoverlayspecification{<+->}



%================packages===============================
%%已有的package:
%hyperref, mathtools, bm
%babel
%amsrefs
\usepackage{float}
\usepackage{multirow}%表格
\usepackage{latexsym}
\usepackage{amsmath,amssymb}
\usepackage{color,xcolor}
\usepackage{algorithm}
\usepackage{amsthm}
\usepackage{graphicx}%图表
\usepackage{wrapfig}%图表
\usepackage{supertabular}
\usepackage{harvard}%citation宏包
\usepackage{xeCJK}


%======================================================
%-----------------------others---------------------------
\newcommand{\eps}{\varepsilon}%简化$\varepsilon$的表示
%================theorem environments==========
%% Note: the following theorem environments
%%       already defined in mathbeamer.cfg
%%       Not need to define it again
%%\theoremstyle{plain}
%%\newtheorem{thm}{Theorem}[section]
%%\newtheorem{lem}[thm]{Lemma}
%%\newtheorem{prop}[thm]{Proposition}
%%\newtheorem{cor}[thm]{Corollary}
%%\theoremstyle{definition}
%%\newtheorem{defn}[thm]{Definition}
%%\theoremstyle{example}
%%\newtheorem{conj}{Conjecture}
%%\newtheorem{exmp}{Example}
%%\newtheorem*{rmk}{Remark}
%%\theoremstyle{break}
%%\newtheorem{bthm}{Theorem}[section]
%%\newtheorem{blem}[thm]{Lemma}
%%\newtheorem{bprop}[thm]{Proposition}
%%\newtheorem{bcor}[thm]{Corollary}
\newcommand{\mthm}{Theorem}
%===================================================
%---------------------------中文字体
\setCJKmainfont[BoldFont={NotoSerifCJKsc-Bold}]{NotoSerifCJKsc-Regular}
\setCJKsansfont{NotoSerifCJKsc-Regular}

%----------------------行距
%\linespread{1.3}

%=========================preferences==========================
%---------------item的符号改为三角形
\setbeamertemplate{items}[triangle]%[circle,ball,]

%----------------citation
\citationstyle{dcu}%-----author,year
%\bibliographystyle{}

%======================at documents===================================
%-------以下命令让幻灯片在每一小节之前都显示一下目录(并突出显示出当前节目录
%\AtBeginSection[]
%{
%	\begin{frame}[shrink]
%	\thispagestyle{empty}
%	\tableofcontents[
%	currentsection,
%	subsectionstyle=shaded/shaded/hide]
%\end{frame}
%}
%\AtBeginSubsection[]
%{
%	\begin{frame}[shrink]
%	\thispagestyle{empty}
%	\addtocounter{framenumber}{-1}
%	\tableofcontents[
%	sectionstyle=show/shaded,
%	subsectionstyle=show/shaded/hide]
%\end{frame}
%}

\AtBeginDocument{%
%------------------目录页
\begin{frame}{
	\sc{Contents}}        %生成目录页,目录太长时加选项[shrink]
%	\setcounter{page}{0}	
\addtocounter{framenumber}{-2}%-----------位置放在beginframe之后,不然无效
\thispagestyle{empty}

\begin{multicols}{2}
	\tableofcontents[hideallsubsections]%,pausesections] 
	% 使目录section一个一个出现的动画效果 [pausesections]
	%列目录时,隐藏所有的小节\tableofcontents[hideallsubsections]
\end{multicols}

\end{frame}
}

\AtEndDocument{%
	%========================
	% thanks
	%========================
	%\section{Thanks}
	\newcounter{ffn}
	\setcounter{ffn}{\value{framenumber}}
	\renewcommand{\insertframenumber}{\inserttotalframenumber}
	\begin{frame}
	\begin{center}
		\LARGE{\bfseries
			Thanks!
		}
	\end{center}
\end{frame}
%========================
% bibliography
%========================
%\section{Reference}
\begin{frame}[t, allowframebreaks]{Reference}
\bibliography{slides/bib}
\end{frame}
\setcounter{framenumber}{\value{ffn}}
}

