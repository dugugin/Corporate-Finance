\usepackage{multirow}%表格
\usepackage{latexsym}
\usepackage{amsmath,amssymb}
\usepackage{color,xcolor}
\usepackage{graphicx}
\usepackage{algorithm}
\usepackage{amsthm}
\usepackage{hyperref}
\usepackage{wrapfig}
%-------------------------超链接颜色---------------------------
%\hypersetup{
%%	bookmarks=true,
%	colorlinks=true,%给超链接文字上色
%	linkcolor=blue,
%	urlcolor=blue,
%	citecolor=true,
%	citecolor=blue,
%%	pdftex,
%	linktocpage=true,   % makes the page number as hyperlink in table of content
%	hyperindex=true
%}


%--------------------------模板风格--------------------------
%\usetheme{AnnArbor}
%\usetheme{PaloAlto}
%\usetheme{Antibes}
\usetheme{CambridgeUS}	%有上下导航
%\usetheme{Copenhagen}
%\usetheme{Darmstadt}
%\usetheme{Dresden}
%\usetheme{Frankfurt}
%\usetheme{Goettingen}
%\usetheme{Hannover}
%\usetheme{Ilmenau}
%\usetheme{JuanLesPins}
%\usetheme{Luebeck}
%\usetheme{Madrid}
%\usetheme{Marburg}
%\usetheme{Montpellier}	%上面多层目录导航
%\usetheme{PaloAlto}
%\usetheme{Singapore}
%\usetheme{Warsaw}


%-------------板式
%\usecolortheme{beaver}		%红色系
%\usecolortheme{default}
%\usecolortheme{crane}		%上导航黄色背景
%\usecolortheme{dolphin}	%上导航蓝色背景
%\usecolortheme{rose}
%\usecolortheme{dove}		%上导航白色背景,灰色割线
\usecolortheme{whale}		%标题深蓝色背景

%---------------------项目符号

\setbeamertemplate{items}[triangle]


%----------------------------option
%\useinnertheme{}
%标题页,列表项目、定理环境、图表环境、脚注default、circles、rectangles、rounded、inmargin

%\useoutertheme{}
%顶部尾部的信息栏、边栏、图标、帧标题等一帧之外的格式。预定义的外部主题有default、infolines、miniframes、smoothbars、sidebar、split、shadow、tree、smoothtree

%\usecolortheme{beaver1}		%改成了蓝色
%default、albatross、beaver、beetle、crane、dolphine、dove、fly、lily、orchid、rose、seagull、seahorse、sidebartab、structure、whale、wolverine


%------------------------------------------------------------
%% Sets the colors for structure and footlines
%\setbeamercolor*{structure}{fg=BayreuthBlue1,bg=white}
%\setbeamercolor*{footline}{fg=black,bg=BayreuthGreen}
%\setbeamercolor*{nofootline}{bg=white}
%
%% Sets the colors of title page and frame titles
%\setbeamercolor*{titlelike}{parent=structure,bg=white}
%\setbeamercolor*{frametitle}{parent=palette primary}
%
%% Sets the colors of the seperation items
%\setbeamercolor*{title separator}{fg=BayreuthBlue1}
%\setbeamercolor*{separation line}{parent=palette primary}
%\setbeamercolor*{fine separation line}{parent=palette primary}
%
%% Sets the colors for normal blocks
%\setbeamercolor*{block title}{parent=palette primary}
%\setbeamercolor*{block body}{parent=normal text,use=block title,bg=block title.bg!20!bg}
%
%% Sets the colors for alert blocks
%\setbeamercolor*{block title alerted}{use=alerted text,fg=white,bg=alerted text.fg!85!black!80!white}
%\setbeamercolor*{block body alerted}{parent=normal text,use=block title alerted,bg=block title alerted.bg!20!bg}
%
%% Sets the colors for example blocks
%\setbeamercolor*{block title example}{parent=palette secondary}
%\setbeamercolor*{block body example}{parent=normal text,use=block title example,bg=block title example.bg!20!bg}
%
%% Sets the colors for footnotes
%\setbeamercolor*{footnote}{fg=BayreuthBlue1}
%\setbeamercolor*{footnote mark}{fg=.}


%--------------------------定理中文化-----------------------
%\newtheorem{prop}{Proposition}
\theoremstyle{definition}
%\theoremstyle{plain}
%\theoremstyle{remark}
\newtheorem*{prop}{命题}%*不自动编号{prop代号}{自动输出label}
\newtheorem*{theo}{定理}
\newtheorem*{ass}{假设}
\newtheorem*{define}{定义}
\newtheorem*{eg}{例}
\renewcommand\figurename{图}
\renewcommand\tablename{表}

%----------------------------字体--------------------
\usefonttheme[onlymath]{serif}
%\usefonttheme{}
%预定义的beamer字体主要包括default、professionalfonts、serif、structurebold、structureitalicserif、structuresmallcapsserif等。
%其中默认字体主题default的效果是整个幻灯片使用无衬线字体,这是多数幻灯片的选择;serif主题则是衬线字体,不过此时最好使用较大的字号和较粗的字体;professionalfonts不对字体有特别的设置,需要使用另外的专门的宏包进行设置;structure开头的几个主题则对beamer中的几个结构有特别设置。
%----------------------设置字体----------------------%
\usepackage{xeCJK}%中文字体
%-----------------------英文字体
%\usefonttheme{professionalfonts}
%\defaultfontfeatures{Scale=MatchLowercase} % 这个参数保证 serif、sans-serif 和 monospace 字体在小写时大小匹配
%\setmainfont[Mapping=tex-text]{Arial} % %Arial设置主字体,使用 XeTeX 的 text-mapping 方案,正确显示 LaTeX 样式的双引号(`` '')
% \setmainfont[Mapping=tex-text]{Palatino Linotype}
%\setsansfont[Mapping=tex-text]{Arial}      %设置衬线字体
%\setsansfont[Mapping=tex-text]{DejaVu Sans}

%\setmonofont{Courier New}                  %设置mono字体
% \setmonofont{Monaco}
%\setmonofont{DejaVu Sans Mono}

%-----------------------设置中文字体
%\setmainfont[BoldFont={timesbd.ttf}]{times.ttf}

\setCJKmainfont[BoldFont={NotoSerifCJKsc-Bold}]{NotoSerifCJKsc-Regular}%衬线字体 缺省中文字体为
%\setCJKsansfont{黑体}
%\setCJKmonofont{仿宋_GB2312}%中文等宽字体
\setCJKsansfont{NotoSerifCJKsc-Regular}

\usepackage{fontspec}
%\newcommand{\song}{\fontspec{NotoSerifCJKsc-Regular.otf}\selectfont}
\newcommand{\hei}{\fontspec{NotoSerifCJKsc-Black.otf}\selectfont}
%\newcommand{\tims}{\fontspec{times.ttf}\selectfont}%times字体
%\newcommand{\gara}{\fontspec{GARA.TTF}\selectfont}%garamond,regular
\newcommand{\garab}{\fontspec{GARABD.TTF}\selectfont}%garamond,bold
\newcommand{\PERTILI}{\fontspec{PERTILI.TTF}\selectfont}%PERTILI.TTF标题细体

%---------------------------title字体
%\usepacage{titlesec}
%\titleformat{\section}{\bfseries}
%\titleformat*{\subsection}{\Large\sectionef}
%\titleformat*{\subsubsection}{\large\bfseries}
%\titleformat{\paragraph}{\bfseries}
%\titleformat*{\subparagraph}{\large\bfseries}

%\titleformat{\chapter}{\centering\Huge\bfseries}{第\,\thechapter\,章}{1em}{}
%其 中, shape 、 before 、 after 参 数 都 被 省 略 掉 了。 format 参 数 将章标题设置为居中( \centering )显示、字号为 \Huge,字体被加粗显示 \bfseries ;在设置 s section 格式,未采用居中,而是采用默认的居左,另外将标题的字号也降了一级( \large )。 label 参数将标题的标签设置为 “第 xxx 章”格式。 sep 参数设置标签与标题内容之间以一个字(1em)的宽度为间隔。以上设置的章标题效果如下图所示:

%------------------frametitle模板
%\setbeamertemplate{frametitle}
%{
%	\begin{centering}
%		\textbf{\insertframetitle}
%		\par
%		\insertframesubtitle
%		\par
%	\end{centering}
%}


%=====================================杂项========================

\XeTeXlinebreaklocale "zh"  % 表示用中文的断行
\XeTeXlinebreakskip = 0pt plus 1pt % 多一点调整的空间
\linespread{1.3}%----------------------行距

%------------------------目录页格式==============
%------------------列目录时,隐藏所有的小节
%\tableofcontents[hideallsubsections]

%----------------以下命令让幻灯片在每一小节之前都显示一下目录(并突出显示出当前节目录)
%\AtBeginSection[]
%{ \begin{frame}
%	\frametitle{目录}
%	\tableofcontents[currentsection,currents section]
%\end{frame}
%\addtocounter{framenumber}{-1}  %目录页不计算页码
%}

%-------------让目录页分栏,这样较多的目录项就可以放在一页上,如下代码如下:
%\usepackage{multicol}
%......
%\begin{frame}
%\frametitle{Contents}
%\begin{multicols}{2}
%	\tableofcontents
%\end{multicols}
%\end{frame}

%-------------------------右下角页码--------------------------
%\setbeamertemplate{footline}[frame number] 
%-------------------在每一子章节前有一张胶片显示目录  
%\AtBeginSection[]  
%{  
%	\begin{frame}<beamer>  
%	\tableofcontents[currentsection]  
%\end{frame}  
%}  
% ------------------设置用acrobat打开就会全屏显示  
%\hypersetup{pdfpagemode=FullScreen}

%--------------------是否逐条显示--------------------
% 除掉以下命令的注解 "%" 后, 许多环境都会自动逐段显示(逐条显示)
%\beamerdefaultoverlayspecification{<+->}

%------------------ transparent for marks-------------
% 设置覆盖的效果,透明  
\setbeamercovered{transparent}  
\usefonttheme[onlysmall]{structurebold} 



