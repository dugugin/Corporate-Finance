
\documentclass[a4paper]{article}
%=========================packages==============================

%=========================packages==============================
%\usepackage{amstext} 
\usepackage{xeCJK}
\usepackage{fontspec}
\usepackage{float}
\usepackage{multirow}%表格
\usepackage{latexsym}
\usepackage{amsmath,amssymb}
\usepackage{color,xcolor}
\usepackage{algorithm}
\usepackage{amsthm}
\usepackage{graphicx}%图表
\usepackage{wrapfig}%图表
%\usepackage{multitoc}%目录分栏,与裸tableofcontents冲突,导致目录不能显示
\usepackage{supertabular}
%\renewcommand{\baselinestretch}{1.2} %行距
\newcommand{\sihao}{\fontsize{14pt}{21pt}\selectfont} 
\newcommand{\wuhao}{\fontsize{12pt}{21pt}\selectfont}
%\newcommand{\sihao}{\fontsize{18pt}{23pt}\selectfont} 
%opening
\usepackage{natbib}
\setcitestyle{authoryear,round,comma,aysep={;},yysep={,},notesep={, }}
\usepackage{geometry}
\usepackage[colorlinks=true,
backref,
linkcolor=blue,
]{hyperref}
%https://en.wikibooks.org/wiki/LaTeX/Hyperlinks#Customization

%===================options====================================
%--------------------目录层级
\setcounter{tocdepth}{1}%只显示section,是全局设置

%-------------------行距、段距
\geometry{left=2cm,right=2cm,top=2cm,bottom=2cm}%scale=0.8%\renewcommand{\baselinestretch}{1.5} \normalsize%不改变字体大小,只改变行距
%\setlength{\baselineskip}{20pt}%行距设为20磅
\setlength{\parskip}{0.7\baselineskip}%段间距=(1+0.5)行间距,/par should be used

%==============================================================
%\vspace{-1.5em}

\AtBeginDocument{
	\maketitle
	 \setcounter{page}{0}
	\thispagestyle{empty}%去掉页码

	\sihao%-----全局四号字体

\tableofcontents
\newpage
}



\AtEndDocument{
\newpage
%==================================================
%\section{Reference}

\bibliographystyle{elsarticle-harv}%elsarticle-harv带url,model2-names无url
%plainnat.bst abbrvnat.bst unsrtnat.bst
\bibliography{bib/cg.bib}
%\href{http://www.nber.org/papers/w6309}{Luigi Zingales. Corporate Governance[J]. NBER, 1997, 52(2): 737-783.}
}

%==============================================================
%\vspace{-1.5em}
\title{Outline\\
\textit{--A Survey of Corporate Governance\\
	A Shleifer, RW Vishny. Journal of Finance, 1996. }
\footnote{\citet{shleifer1996} ,标红字体是没看懂的内容}
}

\author{\vspace{1em}
	{\Large Chins}\\
	Wuhan University
}
\date{\today}

\begin{document}


\section{Introduction}
	这篇论文调查了与公司治理相关的文献,特别关注了世界范围内各种公司治理模式中的投资者法律保护和集中的所有权形式。\par 

	\begin{itemize}
		\item 本文叙述了公司治理的框架。公司治理就是处理因所有权和控制权分离而产生的代理问题,公司治理一个最基本的问题是如何保证投资者能从投资中获取收益。
		\item 描述了几个公司治理的方法:\par
		投资者基于公司的口碑去投资;或者依赖于投资者的乐观情绪;法律保护;集中所有权。
		\item 大投资者为公司治理带来好处的同时也会产生成本:尽管大投资者能解决部分代理问题,但他也会侵占其他投资者利益,产生第二类代理问题。
	\end{itemize}

\section{Background}
	\begin{itemize}
		\item 什么是公司治理\par
		公司治理是可以通过政治途径改革的经济和法律制度。
		\item 为什么投资者需要回报,为什么存在公司治理\par
		现实生活中由于投资是专有的、沉没的,因此产品市场不是完全有效的,代理成本无法被企业最小化,投资者需要确保他们能从投资中得到某种补偿,公司治理机制提供了一些保障。\par
		产品市场竞争不能阻止经理挪用竞争收益,仅仅靠市场竞争不能解决这个问题,因此文章调查了其他解决方法。
		\item 现存治理机制\par
		关于现存英美、日德的公司治理模式有很多争论,尽管最发达经济体的公司治理机制较好地处理了某些方面的代理问题,但它们也并不是完美的。
		
	\end{itemize}

\section{Main story}	
	\begin{itemize}
		\item 首先介绍了代理问题的基本性质(剩余控制权的分配)及一种可能的解决办法(激励合同)
		\item 2-5节介绍了代理问题存在的条件下,公司获取融资的几种方式:\par
		在完全不给投资者任何实际权力时:声誉建设、投资者乐观情绪(但这不能完全解释投资者信任企业并对它投资的原因,因为信息不对称、投资者学习教训)\par
		两个CG最普遍使用的方法:法律保护(eg少数股东保护,严禁经理挪用、假公济私
		),集中所有权(令有效的控制权与有效的现金流权相匹配)。都需要给与投资者一定权力。虽然大投资者仍然依赖于法律体系,但他们在保护自己的利益时,比小投资者更少地依赖法律权利。
		\item 6节谈了许多被广泛使用的公司治理机制的实例,实例解释了法律保护和集中的所有权在公司治理机制中的作用。\par 
		用债务治理、股权治理来作为解决代理问题的替代方法;\par
		用杠杆收购的例子解释所有权集中的好处和坏处;\par
		用国有企业形式作为反例,描述一种失败的公司治理机制
		\item 最后综合3-6节,研究哪种公司治理机制是最好的。\par 
		最有效的那些公司治理模式是法律保护和集中所有权不同的组合模式\par
		但是现有证据并不能证明美国、日德哪种治理模式是最好的。但是纵观一些落后的发展中国家、转型国家,甚至一些富裕的欧洲国家,对他们来说公司治理机制的改革最根本的是要为投资者提供法律保护,至少一部分投资者能受到有效的法律保护,这样能使外延的金融机制起到有效的作用。
	\end{itemize}
	
	

	
\end{document}
