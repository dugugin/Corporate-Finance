
\documentclass[a4paper]{article}
%=========================notedef==============================

%=========================packages==============================
%\usepackage{amstext} 
\usepackage{xeCJK}
\usepackage{fontspec}
\usepackage{float}
\usepackage{multirow}%表格
\usepackage{latexsym}
\usepackage{amsmath,amssymb}
\usepackage{color,xcolor}
\usepackage{algorithm}
\usepackage{amsthm}
\usepackage{graphicx}%图表
\usepackage{wrapfig}%图表
%\usepackage{multitoc}%目录分栏,与裸tableofcontents冲突,导致目录不能显示
\usepackage{supertabular}
%\renewcommand{\baselinestretch}{1.2} %行距
\newcommand{\sihao}{\fontsize{14pt}{21pt}\selectfont} 
\newcommand{\wuhao}{\fontsize{12pt}{21pt}\selectfont}
%\newcommand{\sihao}{\fontsize{18pt}{23pt}\selectfont} 
%opening
\usepackage{natbib}
\setcitestyle{authoryear,round,comma,aysep={;},yysep={,},notesep={, }}
\usepackage{geometry}
\usepackage[colorlinks=true,
backref,
linkcolor=blue,
]{hyperref}
%https://en.wikibooks.org/wiki/LaTeX/Hyperlinks#Customization

%===================options====================================
%--------------------目录层级
\setcounter{tocdepth}{1}%只显示section,是全局设置

%-------------------行距、段距
\geometry{left=2cm,right=2cm,top=2cm,bottom=2cm}%scale=0.8%\renewcommand{\baselinestretch}{1.5} \normalsize%不改变字体大小,只改变行距
%\setlength{\baselineskip}{20pt}%行距设为20磅
\setlength{\parskip}{0.7\baselineskip}%段间距=(1+0.5)行间距,/par should be used

%==============================================================
%\vspace{-1.5em}

\AtBeginDocument{
	\maketitle
	 \setcounter{page}{0}
	\thispagestyle{empty}%去掉页码

	\sihao%-----全局四号字体

\tableofcontents
\newpage
}



\AtEndDocument{
\newpage
%==================================================
%\section{Reference}

\bibliographystyle{elsarticle-harv}%elsarticle-harv带url,model2-names无url
%plainnat.bst abbrvnat.bst unsrtnat.bst
\bibliography{bib/cg.bib}
%\href{http://www.nber.org/papers/w6309}{Luigi Zingales. Corporate Governance[J]. NBER, 1997, 52(2): 737-783.}
}


%==========================title===============================
%\vspace{-1.5em}
\title{Outline\\
\textit{	--Corporate Governance\\	
Luigi Zingales, 1997. }
\footnote{\citet{zingales1997},标红字体是没看懂的内容}
}

\author{\vspace{1em}
	{\Large Chins}\\
	Wuhan University
}
\date{\today}


\begin{document}


\section{Introduction}
Questions
\begin{itemize}
	\item 什么是CG?为什么存在CG问题?为什么亚当斯密所说的“看不见的手”不能自动解决这个问题?
	\item takeovers, financial reconstructing, institutional investors 在公司治理中起到什么作用?
	\item 本文提供了一个系统性的解答,将CG与FIRM THEORY的关键联系明确化。
\end{itemize}
文章结构
\begin{itemize}
	\item 1节告诉我们为什么会需要公司治理系统
	\item 2-3节告诉我们什么是CG,公司的定义是什么;公司治理存在的前提条件
	\item 4节讨论了公司治理的作用,告诉我们一个好的公司治理的三个主要目标
	\item 5节讨论公司治理其中一个目标:控制权的分配。另外两个目标在另两篇文献有讨论
	\item 6-7节讨论不完全契约理论在公司治理中的内涵,及其局限性
\end{itemize}


\section{我们什么时候需要公司治理系统·基本假设}
\begin{itemize}
	\item free-mkt economy在实际不满足
	\item 购买机器的订单--制作机器--市场价格变动--产生了quasi-rent--产生了事后的bargaining--亚当斯密看不见的手对ex-post的surplus的分配无能为力
	\item contract incompleteness also creates room for bargaining.
	\item 公司治理存在的前提:买卖关系之间产生某种quasi-rent,否则竞争市场不会使bargain产生;且quasi-rent不可能在事前被完美地分配,否则也不会有bargain的余地。
\end{itemize}

%
\section{什么是CG}	
\subsection{什么是公司治理}
\begin{description}
	\item[Corporate Governance] 公司治理就是对某个组织形式(corporate: transaction, club, any economic organization)的治理
	\item[Essence of Governance] Bargaining over the ex-post rents
	\item[firm] 公司是建立在经济基础上的法人,具有法律追诉能力\par 
	\item[Def og CG] complex set of constraints that shape the ex-post bargaining over the quasi-rents generated by the firm.
\end{description}

\subsection{firm两个主要的定义和一个更宽泛的定义}
\begin{description}	
	\item[\citet{alchian1972}] 企业是一系列契约的联结
	\item[\citet{grossman1986}] 企业是联合拥有的物质资产的集合
	\item[Rajan\&Zingales(1997,\citeyear{rajan1998})] 一系列特种投资的联结:互相特殊化的资产和人的结合(工人,生产者,消费者)

\end{description}

%
\section{不完全契约及治理·公司治理存在的前提}
公司治理存在的前提条件是
\begin{itemize}
	\item 在事前订立能跟随未来一些变量而改变的合同是有成本的
	\item 合同是不完全的
\end{itemize}

%
\section{公司治理的作用}
公司治理的作用是
\begin{description}
	\item[对事前激励作用] 激励合同\par
	事后对剩余的分割会影响事前的行为(努力成本、私利)
	\item[对无效率的讨价还价] 
	通过对事后讨价还价的效率的影响,最终影响到总价值(eg信息不对称=搭便车、协调成本=控制权分配、流动性约束的程度\citet{aghion1992})
	\item[对风险厌恶] CG通过风险程度、分布,影响总剩余的\textbf{事前}价值。\par
	公司治理使最risk-tolerant的一方去\textbf{承担风险};治理系统的设计会产生不同风险;
	
\end{description}
公司治理的目标是
\begin{itemize}
	\item 最大化增加价值的投资动机,最小化权利寻租的动机
	\item 最小化事后无效的讨价还价
	\item 最小化治理系统风险、将剩余风险分配给最不厌恶风险的人
\end{itemize}


\section{剩余控制权为什么分配给投资者}
重点从激励(公司治理第一个目标)方面讨论控制权的分配问题,控制权应该分配给股东\par
% 
\%契约视角:股东与企业契约不完全,契约得到的保障也不完全,需要以控制权去确保\par 
%
\%GHM不完全契约范围之外:其他资产持有者能更好地通过契约保护自己的投资;其他资产持有者有更多的事后议价能力\par 

\citet{rajan1998} 认为应该将剩余控制权分配给代理人,而且是需要保护自己的投资不受挪用、却对资产应该怎样使用只有少量控制权的代理人。有2个原因
\begin{itemize}
	\item 人力资本不可能在事前被契约化或授权给别人,剩余控制权给他们会导致谋私利
	\item 剩余控制权分配给资金的提供者,能保护投资者\textbf{投资积极性},补偿了他们将资产的特化授权出去的\textbf{沉没成本}。
\end{itemize}

因此在一个更广泛的公司的定义下,公司治理系统中最优的做法是,将剩余控制权分配给资金的提供者,资金提供者将特化资产的权利(right to specialize the asset)授权给第三方,第三方没有资产的特化产生的机会成本,资本的沉没性不会对他们的边际激励产生负面影响。但这个第三方必须为整个公司所有投资的联结负责,而不仅是股东。\par

\citet{blair1999}阐述美国的公司法将这一权利授予了董事会。\par


\section{不完全合约在公司治理中的规范内涵和局限性}
\subsection{内涵}
只有在契约不完全的世界,才有了投机的余地(治理存在的必要)、\\
才有法律保护的意义(立法执法,影响了quasi-rents的分配)、\\
才有政府干预的必要(监管,改善事后的效率)。

\subsection{局限性}
{\color{red}{
a. 不完全契约对所有权分配的解释严重依赖于我们可以缔结什么合同\\
b. 严重依赖于代理人对所有未来意外事故的预测}
}


\section{sum}%sum用中文写
\begin{itemize}
	\item 在一个所有意外情况都可以事前被完全考虑到的世界里,不存在治理的余地。
	\item 在一个不完全契约的世界,公司治理被定义为一系列规范事后
	对公司的剩余收益讨价还价的条件。
	\item 一个治理系统在事前和事后都能起到有效作用,对风险的水平和分布也有影响。
	\item 不完全合约理论能够很好的解释企业家的企业的CG,可以解释所有权如何分配、资本结构如何选择。但它很难解释大的公众公司。
\end{itemize}


\section*{ps}
\begin{tabular}{ll}
	提出了CG		& Berle and Means (1932) \\
	提出了剩余所有权& \citet{grossman1986}\\
	
\end{tabular}


\end{document}
